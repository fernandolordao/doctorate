\documentclass[]{article}
\usepackage{lmodern}
\usepackage{amssymb,amsmath}
\usepackage{ifxetex,ifluatex}
\usepackage{fixltx2e} % provides \textsubscript
\ifnum 0\ifxetex 1\fi\ifluatex 1\fi=0 % if pdftex
  \usepackage[T1]{fontenc}
  \usepackage[utf8]{inputenc}
\else % if luatex or xelatex
  \ifxetex
    \usepackage{mathspec}
  \else
    \usepackage{fontspec}
  \fi
  \defaultfontfeatures{Ligatures=TeX,Scale=MatchLowercase}
\fi
% use upquote if available, for straight quotes in verbatim environments
\IfFileExists{upquote.sty}{\usepackage{upquote}}{}
% use microtype if available
\IfFileExists{microtype.sty}{%
\usepackage{microtype}
\UseMicrotypeSet[protrusion]{basicmath} % disable protrusion for tt fonts
}{}
\usepackage[margin=1in]{geometry}
\usepackage{hyperref}
\hypersetup{unicode=true,
            pdftitle={lab3-ckp1},
            pdfauthor={Fernando Lordao},
            pdfborder={0 0 0},
            breaklinks=true}
\urlstyle{same}  % don't use monospace font for urls
\usepackage{color}
\usepackage{fancyvrb}
\newcommand{\VerbBar}{|}
\newcommand{\VERB}{\Verb[commandchars=\\\{\}]}
\DefineVerbatimEnvironment{Highlighting}{Verbatim}{commandchars=\\\{\}}
% Add ',fontsize=\small' for more characters per line
\usepackage{framed}
\definecolor{shadecolor}{RGB}{248,248,248}
\newenvironment{Shaded}{\begin{snugshade}}{\end{snugshade}}
\newcommand{\KeywordTok}[1]{\textcolor[rgb]{0.13,0.29,0.53}{\textbf{#1}}}
\newcommand{\DataTypeTok}[1]{\textcolor[rgb]{0.13,0.29,0.53}{#1}}
\newcommand{\DecValTok}[1]{\textcolor[rgb]{0.00,0.00,0.81}{#1}}
\newcommand{\BaseNTok}[1]{\textcolor[rgb]{0.00,0.00,0.81}{#1}}
\newcommand{\FloatTok}[1]{\textcolor[rgb]{0.00,0.00,0.81}{#1}}
\newcommand{\ConstantTok}[1]{\textcolor[rgb]{0.00,0.00,0.00}{#1}}
\newcommand{\CharTok}[1]{\textcolor[rgb]{0.31,0.60,0.02}{#1}}
\newcommand{\SpecialCharTok}[1]{\textcolor[rgb]{0.00,0.00,0.00}{#1}}
\newcommand{\StringTok}[1]{\textcolor[rgb]{0.31,0.60,0.02}{#1}}
\newcommand{\VerbatimStringTok}[1]{\textcolor[rgb]{0.31,0.60,0.02}{#1}}
\newcommand{\SpecialStringTok}[1]{\textcolor[rgb]{0.31,0.60,0.02}{#1}}
\newcommand{\ImportTok}[1]{#1}
\newcommand{\CommentTok}[1]{\textcolor[rgb]{0.56,0.35,0.01}{\textit{#1}}}
\newcommand{\DocumentationTok}[1]{\textcolor[rgb]{0.56,0.35,0.01}{\textbf{\textit{#1}}}}
\newcommand{\AnnotationTok}[1]{\textcolor[rgb]{0.56,0.35,0.01}{\textbf{\textit{#1}}}}
\newcommand{\CommentVarTok}[1]{\textcolor[rgb]{0.56,0.35,0.01}{\textbf{\textit{#1}}}}
\newcommand{\OtherTok}[1]{\textcolor[rgb]{0.56,0.35,0.01}{#1}}
\newcommand{\FunctionTok}[1]{\textcolor[rgb]{0.00,0.00,0.00}{#1}}
\newcommand{\VariableTok}[1]{\textcolor[rgb]{0.00,0.00,0.00}{#1}}
\newcommand{\ControlFlowTok}[1]{\textcolor[rgb]{0.13,0.29,0.53}{\textbf{#1}}}
\newcommand{\OperatorTok}[1]{\textcolor[rgb]{0.81,0.36,0.00}{\textbf{#1}}}
\newcommand{\BuiltInTok}[1]{#1}
\newcommand{\ExtensionTok}[1]{#1}
\newcommand{\PreprocessorTok}[1]{\textcolor[rgb]{0.56,0.35,0.01}{\textit{#1}}}
\newcommand{\AttributeTok}[1]{\textcolor[rgb]{0.77,0.63,0.00}{#1}}
\newcommand{\RegionMarkerTok}[1]{#1}
\newcommand{\InformationTok}[1]{\textcolor[rgb]{0.56,0.35,0.01}{\textbf{\textit{#1}}}}
\newcommand{\WarningTok}[1]{\textcolor[rgb]{0.56,0.35,0.01}{\textbf{\textit{#1}}}}
\newcommand{\AlertTok}[1]{\textcolor[rgb]{0.94,0.16,0.16}{#1}}
\newcommand{\ErrorTok}[1]{\textcolor[rgb]{0.64,0.00,0.00}{\textbf{#1}}}
\newcommand{\NormalTok}[1]{#1}
\usepackage{graphicx,grffile}
\makeatletter
\def\maxwidth{\ifdim\Gin@nat@width>\linewidth\linewidth\else\Gin@nat@width\fi}
\def\maxheight{\ifdim\Gin@nat@height>\textheight\textheight\else\Gin@nat@height\fi}
\makeatother
% Scale images if necessary, so that they will not overflow the page
% margins by default, and it is still possible to overwrite the defaults
% using explicit options in \includegraphics[width, height, ...]{}
\setkeys{Gin}{width=\maxwidth,height=\maxheight,keepaspectratio}
\IfFileExists{parskip.sty}{%
\usepackage{parskip}
}{% else
\setlength{\parindent}{0pt}
\setlength{\parskip}{6pt plus 2pt minus 1pt}
}
\setlength{\emergencystretch}{3em}  % prevent overfull lines
\providecommand{\tightlist}{%
  \setlength{\itemsep}{0pt}\setlength{\parskip}{0pt}}
\setcounter{secnumdepth}{0}
% Redefines (sub)paragraphs to behave more like sections
\ifx\paragraph\undefined\else
\let\oldparagraph\paragraph
\renewcommand{\paragraph}[1]{\oldparagraph{#1}\mbox{}}
\fi
\ifx\subparagraph\undefined\else
\let\oldsubparagraph\subparagraph
\renewcommand{\subparagraph}[1]{\oldsubparagraph{#1}\mbox{}}
\fi

%%% Use protect on footnotes to avoid problems with footnotes in titles
\let\rmarkdownfootnote\footnote%
\def\footnote{\protect\rmarkdownfootnote}

%%% Change title format to be more compact
\usepackage{titling}

% Create subtitle command for use in maketitle
\newcommand{\subtitle}[1]{
  \posttitle{
    \begin{center}\large#1\end{center}
    }
}

\setlength{\droptitle}{-2em}
  \title{lab3-ckp1}
  \pretitle{\vspace{\droptitle}\centering\huge}
  \posttitle{\par}
  \author{Fernando Lordao}
  \preauthor{\centering\large\emph}
  \postauthor{\par}
  \predate{\centering\large\emph}
  \postdate{\par}
  \date{16 de maio de 2018}


\begin{document}
\maketitle

\begin{Shaded}
\begin{Highlighting}[]
\CommentTok{#running_mode = "finalWeb"}
\NormalTok{running_mode =}\StringTok{ "reducedLocal"}
\CommentTok{#running_mode = "finalLocal"}

\ControlFlowTok{if}\NormalTok{(running_mode }\OperatorTok{==}\StringTok{ "finalWeb"}\NormalTok{) \{}
\NormalTok{    path_to_read =}\StringTok{ "https://github.com/wikimedia-research/Discovery-Hiring-Analyst-2016/raw/master/events_log.csv.gz"}
\NormalTok{    lines_to_read =}\StringTok{ }\OperatorTok{-}\DecValTok{1}
\NormalTok{\} }\ControlFlowTok{else} \ControlFlowTok{if}\NormalTok{(running_mode }\OperatorTok{==}\StringTok{ "finalLocal"}\NormalTok{) \{}
\NormalTok{    path_to_read =}\StringTok{ "data/events_log.csv"}
\NormalTok{    lines_to_read =}\StringTok{ }\OperatorTok{-}\DecValTok{1}
\NormalTok{\} }\ControlFlowTok{else} \ControlFlowTok{if}\NormalTok{(running_mode }\OperatorTok{==}\StringTok{ "reducedLocal"}\NormalTok{) \{}
\NormalTok{    path_to_read =}\StringTok{ "data/events_log.csv"}
\NormalTok{    lines_to_read =}\StringTok{ }\DecValTok{5000}
\NormalTok{\} }\ControlFlowTok{else}\NormalTok{ \{}
    \KeywordTok{message}\NormalTok{(}\StringTok{"Running mode not specified."}\NormalTok{)}
    \ControlFlowTok{break}\NormalTok{() }\CommentTok{#encontrar a função certa para finalizar o script.}
\NormalTok{\}}

\KeywordTok{set.seed}\NormalTok{(}\DecValTok{20180520}\NormalTok{)}
\end{Highlighting}
\end{Shaded}

\begin{Shaded}
\begin{Highlighting}[]
\KeywordTok{library}\NormalTok{(tidyverse)}
\end{Highlighting}
\end{Shaded}

\begin{verbatim}
## -- Attaching packages ------------------------------------------------------ tidyverse 1.2.1 --
\end{verbatim}

\begin{verbatim}
## v ggplot2 2.2.1     v purrr   0.2.4
## v tibble  1.4.2     v dplyr   0.7.4
## v tidyr   0.8.0     v stringr 1.3.0
## v readr   1.1.1     v forcats 0.3.0
\end{verbatim}

\begin{verbatim}
## -- Conflicts --------------------------------------------------------- tidyverse_conflicts() --
## x dplyr::filter() masks stats::filter()
## x dplyr::lag()    masks stats::lag()
\end{verbatim}

\begin{Shaded}
\begin{Highlighting}[]
\KeywordTok{library}\NormalTok{(lubridate)}
\end{Highlighting}
\end{Shaded}

\begin{verbatim}
## 
## Attaching package: 'lubridate'
\end{verbatim}

\begin{verbatim}
## The following object is masked from 'package:base':
## 
##     date
\end{verbatim}

\begin{Shaded}
\begin{Highlighting}[]
\KeywordTok{library}\NormalTok{(chron)}
\end{Highlighting}
\end{Shaded}

\begin{verbatim}
## 
## Attaching package: 'chron'
\end{verbatim}

\begin{verbatim}
## The following objects are masked from 'package:lubridate':
## 
##     days, hours, minutes, seconds, years
\end{verbatim}

\begin{Shaded}
\begin{Highlighting}[]
\KeywordTok{library}\NormalTok{(boot)}
\KeywordTok{theme_set}\NormalTok{(}\KeywordTok{theme_bw}\NormalTok{())}
\end{Highlighting}
\end{Shaded}

\subsection{Descrição geral}\label{descricao-geral}

Neste documento vamos explorar os dados disponibilizados pela Wikimedia
e tentar, no percurso dessa análise exploratória, responder às questões
levantadas fazendo inferências sobre a população através de bootstraps.

\subsection{Carga do log de eventos}\label{carga-do-log-de-eventos}

\begin{Shaded}
\begin{Highlighting}[]
\NormalTok{events =}\StringTok{ }\KeywordTok{read_csv}\NormalTok{(path_to_read) }\OperatorTok\StringTok{ }
\StringTok{    }\KeywordTok{head}\NormalTok{(lines_to_read)}
\end{Highlighting}
\end{Shaded}

\begin{verbatim}
## Parsed with column specification:
## cols(
##   uuid = col_character(),
##   timestamp = col_double(),
##   session_id = col_character(),
##   group = col_character(),
##   action = col_character(),
##   checkin = col_integer(),
##   page_id = col_character(),
##   n_results = col_integer(),
##   result_position = col_integer()
## )
\end{verbatim}

\begin{Shaded}
\begin{Highlighting}[]
\CommentTok{#ATTENTION! Não entendi o uso do slice}
\CommentTok{#events = events %>% slice(1:5e4) # Útil para testar código em dados pequenos. Comente na hora de processá-los para valer.}

\NormalTok{events =}\StringTok{ }\NormalTok{events }\OperatorTok\StringTok{ }
\StringTok{    }\KeywordTok{mutate}\NormalTok{(}
        \DataTypeTok{date =} \KeywordTok{round_date}\NormalTok{(}\KeywordTok{ymd_hms}\NormalTok{(timestamp), }\DataTypeTok{unit =} \StringTok{"day"}\NormalTok{),}
        \DataTypeTok{date_week =} \KeywordTok{paste}\NormalTok{(date, }\KeywordTok{weekdays}\NormalTok{(date, }\DataTypeTok{abbreviate =} \OtherTok{TRUE}\NormalTok{))}
\NormalTok{    )}
\end{Highlighting}
\end{Shaded}

\subsection{PERGUNTA 1:}\label{pergunta-1}

\subsection{Qual a taxa geral de cliques durante o dia? Como ela varia
entre os
grupos?}\label{qual-a-taxa-geral-de-cliques-durante-o-dia-como-ela-varia-entre-os-grupos}

Conforme descrição do problema pela Wikimedia, a taxa de cliques é
definida da seguinte maneira:

\emph{clickthrough rate: the proportion of search sessions where the
user clicked on one of the results displayed}

Para isso é preciso criar novas variáveis que permitam identificar as
buscas e também as visitas dentre de cada busca. Faremos isso usando o
código a seguir.

\begin{Shaded}
\begin{Highlighting}[]
\NormalTok{events =}\StringTok{ }\NormalTok{events }\OperatorTok\StringTok{ }
\StringTok{    }\KeywordTok{group_by}\NormalTok{(session_id) }\OperatorTok\StringTok{ }
\StringTok{    }\KeywordTok{arrange}\NormalTok{(timestamp) }\OperatorTok\StringTok{ }
\StringTok{    }\KeywordTok{mutate}\NormalTok{(}
        \DataTypeTok{search_index =} \KeywordTok{cumsum}\NormalTok{(action }\OperatorTok{==}\StringTok{ "searchResultPage"}\NormalTok{) }\CommentTok{# Sequenciador de buscas realizadas dentro de cada sessão.}
\NormalTok{    )}

\NormalTok{events =}\StringTok{ }\NormalTok{events }\OperatorTok\StringTok{ }
\StringTok{    }\KeywordTok{group_by}\NormalTok{(session_id, search_index) }\OperatorTok\StringTok{ }
\StringTok{    }\KeywordTok{arrange}\NormalTok{(timestamp) }\OperatorTok\StringTok{ }
\StringTok{    }\KeywordTok{mutate}\NormalTok{(}
        \DataTypeTok{visit_index =} \KeywordTok{cumsum}\NormalTok{(action }\OperatorTok{==}\StringTok{ "visitPage"}\NormalTok{) }\CommentTok{# Sequenciador de visitas realizadas dentro de cada resultado de pesquisa.}
\NormalTok{    )}
\end{Highlighting}
\end{Shaded}

Segue código para produzir os dados diários, por grupo, da
\emph{clickthrough rate} considerando a noção definida pela Wikimedia.

\begin{Shaded}
\begin{Highlighting}[]
\NormalTok{events_q1 =}\StringTok{ }\NormalTok{events }\OperatorTok
\StringTok{  }\KeywordTok{group_by}\NormalTok{(group, date_week, session_id, search_index) }\OperatorTok\StringTok{ }
\StringTok{  }\KeywordTok{summarize}\NormalTok{(}\DataTypeTok{visit_count =} \KeywordTok{max}\NormalTok{(visit_index)) }\OperatorTok\StringTok{ }
\StringTok{  }\KeywordTok{group_by}\NormalTok{(group, date_week) }\OperatorTok\StringTok{ }
\StringTok{  }\KeywordTok{summarize}\NormalTok{(}
    \DataTypeTok{nonzero_click =} \KeywordTok{sum}\NormalTok{(visit_count}\OperatorTok{>}\DecValTok{0}\NormalTok{),}
    \DataTypeTok{zero_click =} \KeywordTok{sum}\NormalTok{(visit_count}\OperatorTok{==}\DecValTok{0}\NormalTok{),}
    \DataTypeTok{clickthroughrate =}\NormalTok{ nonzero_click }\OperatorTok{/}\StringTok{ }\NormalTok{(nonzero_click }\OperatorTok{+}\StringTok{ }\NormalTok{zero_click)}
\NormalTok{  )}


\NormalTok{events_q1 }\OperatorTok\StringTok{ }
\StringTok{    }\KeywordTok{ggplot}\NormalTok{(}\KeywordTok{aes}\NormalTok{(}\DataTypeTok{x =}\NormalTok{ date_week, }\DataTypeTok{y =}\NormalTok{ clickthroughrate}\OperatorTok{*}\DecValTok{100}\NormalTok{, }\DataTypeTok{fill =}\NormalTok{ group)) }\OperatorTok{+}
\StringTok{    }\KeywordTok{geom_col}\NormalTok{(}\DataTypeTok{position =} \StringTok{"dodge"}\NormalTok{) }\OperatorTok{+}
\StringTok{    }\KeywordTok{labs}\NormalTok{(}\DataTypeTok{x =} \StringTok{"Dias"}\NormalTok{, }\DataTypeTok{y =} \StringTok{"Taxa de cliques (%)"}\NormalTok{, }\DataTypeTok{fill =} \StringTok{"Grupos"}\NormalTok{, }\DataTypeTok{title =} \StringTok{"Taxa geral de cliques por dia e por grupo"}\NormalTok{) }\OperatorTok{+}
\StringTok{    }\KeywordTok{theme}\NormalTok{(}\DataTypeTok{axis.text.x =} \KeywordTok{element_text}\NormalTok{(}\DataTypeTok{angle=}\DecValTok{60}\NormalTok{, }\DataTypeTok{hjust=}\DecValTok{1}\NormalTok{))}
\end{Highlighting}
\end{Shaded}

\includegraphics{lab3-ckp1_files/figure-latex/unnamed-chunk-2-1.pdf}

\subsection{Analisando a média de taxa de cliques diária geral (sem
distinção de
grupos)}\label{analisando-a-media-de-taxa-de-cliques-diaria-geral-sem-distincao-de-grupos}

\begin{Shaded}
\begin{Highlighting}[]
\NormalTok{funcao_bootstrap_q1_global <-}\StringTok{ }\ControlFlowTok{function}\NormalTok{(data, indexes)\{}
\NormalTok{  daily_rate =}\StringTok{ }\NormalTok{data }\OperatorTok\StringTok{ }
\StringTok{    }\KeywordTok{slice}\NormalTok{(indexes) }\OperatorTok\StringTok{ }
\StringTok{    }\KeywordTok{group_by}\NormalTok{(date_week) }\OperatorTok\StringTok{ }
\StringTok{    }\KeywordTok{summarize}\NormalTok{(}
      \DataTypeTok{clickthroughrate =}\NormalTok{ (}\KeywordTok{sum}\NormalTok{(click_count}\OperatorTok{>}\DecValTok{0}\NormalTok{) }\OperatorTok{/}\StringTok{ }\KeywordTok{sum}\NormalTok{(click_count}\OperatorTok{>=}\DecValTok{0}\NormalTok{)) }\OperatorTok{*}\StringTok{ }\DecValTok{100}
\NormalTok{    )}
  \KeywordTok{return}\NormalTok{(}
    \KeywordTok{mean}\NormalTok{(daily_rate}\OperatorTok{$}\NormalTok{clickthroughrate)}
\NormalTok{  )}
\NormalTok{\}}

\NormalTok{events_q1b =}\StringTok{ }\NormalTok{events }\OperatorTok\StringTok{ }
\StringTok{    }\KeywordTok{group_by}\NormalTok{(date_week, session_id, search_index) }\OperatorTok\StringTok{ }
\StringTok{    }\KeywordTok{mutate}\NormalTok{(}\DataTypeTok{click_count =} \KeywordTok{max}\NormalTok{(visit_index)) }\OperatorTok\StringTok{ }
\StringTok{    }\KeywordTok{subset}\NormalTok{(action }\OperatorTok{==}\StringTok{ "searchResultPage"}\NormalTok{, }\DataTypeTok{select =} \KeywordTok{c}\NormalTok{(}\StringTok{"date_week"}\NormalTok{, }\StringTok{"click_count"}\NormalTok{))}


\NormalTok{bootstraps_q1 <-}\StringTok{ }\KeywordTok{boot}\NormalTok{(}\DataTypeTok{data =}\NormalTok{ events_q1b, }
                   \DataTypeTok{statistic =}\NormalTok{ funcao_bootstrap_q1_global, }\CommentTok{# <- referência para a função }
                   \DataTypeTok{R =} \DecValTok{1000}\NormalTok{) }\CommentTok{# número de bootstraps}

\KeywordTok{glimpse}\NormalTok{(bootstraps_q1}\OperatorTok{$}\NormalTok{t)}
\end{Highlighting}
\end{Shaded}

\begin{verbatim}
##  num [1:1000, 1] 0.651 0.775 1.069 0.46 1.125 ...
\end{verbatim}

Vamos ver como ficou a distribuição amostral do parâmetro populacional
estimado via bootstrap. Estão marcadas no histograma duas linhas
verticais para os quantis 2,5\% e 97,5\%.

\begin{Shaded}
\begin{Highlighting}[]
\KeywordTok{tibble}\NormalTok{(}\DataTypeTok{parametro =} \KeywordTok{as.double}\NormalTok{(bootstraps_q1}\OperatorTok{$}\NormalTok{t)) }\OperatorTok\StringTok{ }
\StringTok{  }\KeywordTok{ggplot}\NormalTok{(}\KeywordTok{aes}\NormalTok{(}\DataTypeTok{x =}\NormalTok{ parametro)) }\OperatorTok{+}\StringTok{ }
\StringTok{  }\KeywordTok{geom_histogram}\NormalTok{(}\DataTypeTok{binwidth =} \FloatTok{0.1}\NormalTok{, }\DataTypeTok{fill =} \StringTok{"white"}\NormalTok{, }\DataTypeTok{color =} \StringTok{"blue"}\NormalTok{)}\OperatorTok{+}
\StringTok{  }\KeywordTok{geom_vline}\NormalTok{(}\DataTypeTok{xintercept =} \KeywordTok{quantile}\NormalTok{(bootstraps_q1}\OperatorTok{$}\NormalTok{t, }\FloatTok{0.025}\NormalTok{)[[}\DecValTok{1}\NormalTok{]]) }\OperatorTok{+}\StringTok{ }
\StringTok{  }\KeywordTok{geom_vline}\NormalTok{(}\DataTypeTok{xintercept =} \KeywordTok{quantile}\NormalTok{(bootstraps_q1}\OperatorTok{$}\NormalTok{t, }\FloatTok{0.975}\NormalTok{)[[}\DecValTok{1}\NormalTok{]])}
\end{Highlighting}
\end{Shaded}

\includegraphics{lab3-ckp1_files/figure-latex/unnamed-chunk-4-1.pdf}

Agora sim, vamos solicitar o cálculo dos intervalos de confiança que
contém o parâmetro calculado em 95\% das vezes que esse cálculo for
realizado.

\begin{Shaded}
\begin{Highlighting}[]
\KeywordTok{boot.ci}\NormalTok{(}\DataTypeTok{boot.out =}\NormalTok{ bootstraps_q1, }\DataTypeTok{conf =} \FloatTok{0.95}\NormalTok{, }\DataTypeTok{type =} \StringTok{"basic"}\NormalTok{)}
\end{Highlighting}
\end{Shaded}

\begin{verbatim}
## BOOTSTRAP CONFIDENCE INTERVAL CALCULATIONS
## Based on 1000 bootstrap replicates
## 
## CALL : 
## boot.ci(boot.out = bootstraps_q1, conf = 0.95, type = "basic")
## 
## Intervals : 
## Level      Basic         
## 95%   ( 0.2662,  1.1095 )  
## Calculations and Intervals on Original Scale
\end{verbatim}

Portanto, baseado nesse bootstrap, podemos afirmar com 95\% de confiança
que a taxa de cliques da população está entre 29,10\% e 29,77\%.

\subsection{Analisando a média de taxa de cliques diária geral (do grupo
A)}\label{analisando-a-media-de-taxa-de-cliques-diaria-geral-do-grupo-a}

\begin{Shaded}
\begin{Highlighting}[]
\NormalTok{events_q1ba =}\StringTok{ }\NormalTok{events }\OperatorTok\StringTok{ }
\StringTok{    }\KeywordTok{filter}\NormalTok{(group }\OperatorTok{==}\StringTok{ "a"}\NormalTok{) }\OperatorTok\StringTok{ }
\StringTok{    }\KeywordTok{group_by}\NormalTok{(date_week, session_id, search_index) }\OperatorTok\StringTok{ }
\StringTok{    }\KeywordTok{mutate}\NormalTok{(}\DataTypeTok{click_count =} \KeywordTok{max}\NormalTok{(visit_index)) }\OperatorTok\StringTok{ }
\StringTok{    }\KeywordTok{subset}\NormalTok{(action }\OperatorTok{==}\StringTok{ "searchResultPage"}\NormalTok{, }\DataTypeTok{select =} \KeywordTok{c}\NormalTok{(}\StringTok{"date_week"}\NormalTok{, }\StringTok{"click_count"}\NormalTok{))}


\NormalTok{bootstraps_q1 <-}\StringTok{ }\KeywordTok{boot}\NormalTok{(}\DataTypeTok{data =}\NormalTok{ events_q1ba, }
                   \DataTypeTok{statistic =}\NormalTok{ funcao_bootstrap_q1_global, }\CommentTok{# <- referência para a função }
                   \DataTypeTok{R =} \DecValTok{1000}\NormalTok{) }\CommentTok{# número de bootstraps}

\KeywordTok{glimpse}\NormalTok{(bootstraps_q1}\OperatorTok{$}\NormalTok{t)}
\end{Highlighting}
\end{Shaded}

\begin{verbatim}
##  num [1:1000, 1] 1.196 1.366 0.987 0.987 1.346 ...
\end{verbatim}

Vamos ver como ficou a distribuição amostral do parâmetro populacional
estimado via bootstrap. Estão marcadas no histograma duas linhas
verticais para os quantis 2,5\% e 97,5\%.

\begin{Shaded}
\begin{Highlighting}[]
\KeywordTok{tibble}\NormalTok{(}\DataTypeTok{parametro =} \KeywordTok{as.double}\NormalTok{(bootstraps_q1}\OperatorTok{$}\NormalTok{t)) }\OperatorTok\StringTok{ }
\StringTok{  }\KeywordTok{ggplot}\NormalTok{(}\KeywordTok{aes}\NormalTok{(}\DataTypeTok{x =}\NormalTok{ parametro)) }\OperatorTok{+}\StringTok{ }
\StringTok{  }\KeywordTok{geom_histogram}\NormalTok{(}\DataTypeTok{binwidth =} \FloatTok{0.1}\NormalTok{, }\DataTypeTok{fill =} \StringTok{"white"}\NormalTok{, }\DataTypeTok{color =} \StringTok{"blue"}\NormalTok{)}\OperatorTok{+}
\StringTok{  }\KeywordTok{geom_vline}\NormalTok{(}\DataTypeTok{xintercept =} \KeywordTok{quantile}\NormalTok{(bootstraps_q1}\OperatorTok{$}\NormalTok{t, }\FloatTok{0.025}\NormalTok{)[[}\DecValTok{1}\NormalTok{]]) }\OperatorTok{+}\StringTok{ }
\StringTok{  }\KeywordTok{geom_vline}\NormalTok{(}\DataTypeTok{xintercept =} \KeywordTok{quantile}\NormalTok{(bootstraps_q1}\OperatorTok{$}\NormalTok{t, }\FloatTok{0.975}\NormalTok{)[[}\DecValTok{1}\NormalTok{]])}
\end{Highlighting}
\end{Shaded}

\includegraphics{lab3-ckp1_files/figure-latex/unnamed-chunk-7-1.pdf}

Agora sim, vamos solicitar o cálculo dos intervalos de confiança que
contém o parâmetro calculado em 95\% das vezes que esse cálculo for
realizado.

\begin{Shaded}
\begin{Highlighting}[]
\KeywordTok{boot.ci}\NormalTok{(}\DataTypeTok{boot.out =}\NormalTok{ bootstraps_q1, }\DataTypeTok{conf =} \FloatTok{0.95}\NormalTok{, }\DataTypeTok{type =} \StringTok{"basic"}\NormalTok{)}
\end{Highlighting}
\end{Shaded}

\begin{verbatim}
## BOOTSTRAP CONFIDENCE INTERVAL CALCULATIONS
## Based on 1000 bootstrap replicates
## 
## CALL : 
## boot.ci(boot.out = bootstraps_q1, conf = 0.95, type = "basic")
## 
## Intervals : 
## Level      Basic         
## 95%   ( 0.394,  1.733 )  
## Calculations and Intervals on Original Scale
\end{verbatim}

Portanto, baseado nesse bootstrap, podemos afirmar com 95\% de confiança
que a taxa de cliques no grupo A da população está entre 28,46\% e
29,06\%.

\subsection{Analisando a média de taxa de cliques diária geral (do grupo
B)}\label{analisando-a-media-de-taxa-de-cliques-diaria-geral-do-grupo-b}

\begin{Shaded}
\begin{Highlighting}[]
\NormalTok{events_q1bb =}\StringTok{ }\NormalTok{events }\OperatorTok\StringTok{ }
\StringTok{    }\KeywordTok{filter}\NormalTok{(group }\OperatorTok{==}\StringTok{ "b"}\NormalTok{) }\OperatorTok\StringTok{ }
\StringTok{    }\KeywordTok{group_by}\NormalTok{(date_week, session_id, search_index) }\OperatorTok\StringTok{ }
\StringTok{    }\KeywordTok{mutate}\NormalTok{(}\DataTypeTok{click_count =} \KeywordTok{max}\NormalTok{(visit_index)) }\OperatorTok\StringTok{ }
\StringTok{    }\KeywordTok{subset}\NormalTok{(action }\OperatorTok{==}\StringTok{ "searchResultPage"}\NormalTok{, }\DataTypeTok{select =} \KeywordTok{c}\NormalTok{(}\StringTok{"date_week"}\NormalTok{, }\StringTok{"click_count"}\NormalTok{))}


\NormalTok{bootstraps_q1 <-}\StringTok{ }\KeywordTok{boot}\NormalTok{(}\DataTypeTok{data =}\NormalTok{ events_q1bb, }
                   \DataTypeTok{statistic =}\NormalTok{ funcao_bootstrap_q1_global, }\CommentTok{# <- referência para a função }
                   \DataTypeTok{R =} \DecValTok{1000}\NormalTok{) }\CommentTok{# número de bootstraps}

\KeywordTok{glimpse}\NormalTok{(bootstraps_q1}\OperatorTok{$}\NormalTok{t)}
\end{Highlighting}
\end{Shaded}

\begin{verbatim}
##  num [1:1000, 1] 0 0 0 0 0 0 0 0 0 0 ...
\end{verbatim}

Vamos ver como ficou a distribuição amostral do parâmetro populacional
estimado via bootstrap. Estão marcadas no histograma duas linhas
verticais para os quantis 2,5\% e 97,5\%.

\begin{Shaded}
\begin{Highlighting}[]
\KeywordTok{tibble}\NormalTok{(}\DataTypeTok{parametro =} \KeywordTok{as.double}\NormalTok{(bootstraps_q1}\OperatorTok{$}\NormalTok{t)) }\OperatorTok\StringTok{ }
\StringTok{  }\KeywordTok{ggplot}\NormalTok{(}\KeywordTok{aes}\NormalTok{(}\DataTypeTok{x =}\NormalTok{ parametro)) }\OperatorTok{+}\StringTok{ }
\StringTok{  }\KeywordTok{geom_histogram}\NormalTok{(}\DataTypeTok{binwidth =} \FloatTok{0.1}\NormalTok{, }\DataTypeTok{fill =} \StringTok{"white"}\NormalTok{, }\DataTypeTok{color =} \StringTok{"blue"}\NormalTok{)}\OperatorTok{+}
\StringTok{  }\KeywordTok{geom_vline}\NormalTok{(}\DataTypeTok{xintercept =} \KeywordTok{quantile}\NormalTok{(bootstraps_q1}\OperatorTok{$}\NormalTok{t, }\FloatTok{0.025}\NormalTok{)[[}\DecValTok{1}\NormalTok{]]) }\OperatorTok{+}\StringTok{ }
\StringTok{  }\KeywordTok{geom_vline}\NormalTok{(}\DataTypeTok{xintercept =} \KeywordTok{quantile}\NormalTok{(bootstraps_q1}\OperatorTok{$}\NormalTok{t, }\FloatTok{0.975}\NormalTok{)[[}\DecValTok{1}\NormalTok{]])}
\end{Highlighting}
\end{Shaded}

\includegraphics{lab3-ckp1_files/figure-latex/unnamed-chunk-10-1.pdf}

Agora sim, vamos solicitar o cálculo dos intervalos de confiança que
contém o parâmetro calculado em 95\% das vezes que esse cálculo for
realizado.

\begin{Shaded}
\begin{Highlighting}[]
\KeywordTok{boot.ci}\NormalTok{(}\DataTypeTok{boot.out =}\NormalTok{ bootstraps_q1, }\DataTypeTok{conf =} \FloatTok{0.95}\NormalTok{, }\DataTypeTok{type =} \StringTok{"basic"}\NormalTok{)}
\end{Highlighting}
\end{Shaded}

\begin{verbatim}
## BOOTSTRAP CONFIDENCE INTERVAL CALCULATIONS
## Based on 1000 bootstrap replicates
## 
## CALL : 
## boot.ci(boot.out = bootstraps_q1, conf = 0.95, type = "basic")
## 
## Intervals : 
## Level      Basic         
## 95%   ( 0,  0 )  
## Calculations and Intervals on Original Scale
\end{verbatim}

Portanto, baseado nesse bootstrap, podemos afirmar com 95\% de confiança
que a taxa de cliques no grupo A da população está entre 14,72\% e
15,36\%.

\subsection{Analisando a diferença da média de taxa de cliques diária
entre o grupo A e o grupo
B}\label{analisando-a-diferenca-da-media-de-taxa-de-cliques-diaria-entre-o-grupo-a-e-o-grupo-b}

\begin{Shaded}
\begin{Highlighting}[]
\NormalTok{events_q1bdif =}\StringTok{ }\NormalTok{events }\OperatorTok\StringTok{ }
\StringTok{    }\KeywordTok{group_by}\NormalTok{(group, date_week, session_id, search_index) }\OperatorTok\StringTok{ }
\StringTok{    }\KeywordTok{mutate}\NormalTok{(}\DataTypeTok{click_count =} \KeywordTok{max}\NormalTok{(visit_index)) }\OperatorTok\StringTok{ }
\StringTok{    }\KeywordTok{subset}\NormalTok{(action }\OperatorTok{==}\StringTok{ "searchResultPage"}\NormalTok{, }\DataTypeTok{select =} \KeywordTok{c}\NormalTok{(}\StringTok{"group"}\NormalTok{, }\StringTok{"date_week"}\NormalTok{, }\StringTok{"click_count"}\NormalTok{))}

\NormalTok{funcao_bootstrap_q1_dif <-}\StringTok{ }\ControlFlowTok{function}\NormalTok{(data, indexes)\{}
\NormalTok{  daily_rate =}\StringTok{ }\NormalTok{data }\OperatorTok\StringTok{ }
\StringTok{    }\KeywordTok{slice}\NormalTok{(indexes) }\OperatorTok\StringTok{ }
\StringTok{    }\KeywordTok{group_by}\NormalTok{(group, date_week) }\OperatorTok\StringTok{ }
\StringTok{    }\KeywordTok{summarize}\NormalTok{(}
      \DataTypeTok{clickthroughrate =}\NormalTok{ (}\KeywordTok{sum}\NormalTok{(click_count}\OperatorTok{>}\DecValTok{0}\NormalTok{) }\OperatorTok{/}\StringTok{ }\KeywordTok{sum}\NormalTok{(click_count}\OperatorTok{>=}\DecValTok{0}\NormalTok{)) }\OperatorTok{*}\StringTok{ }\DecValTok{100}
\NormalTok{    )}
\NormalTok{  mean_rate_a =}\StringTok{ }\NormalTok{daily_rate }\OperatorTok\StringTok{ }
\StringTok{    }\KeywordTok{filter}\NormalTok{(group }\OperatorTok{==}\StringTok{ "a"}\NormalTok{) }\OperatorTok\StringTok{ }
\StringTok{    }\KeywordTok{summarize}\NormalTok{(}
      \DataTypeTok{meanrate =} \KeywordTok{mean}\NormalTok{(clickthroughrate)}
\NormalTok{    ) }\OperatorTok\StringTok{ }
\StringTok{    }\KeywordTok{pull}\NormalTok{ (meanrate)}
    
\NormalTok{  mean_rate_b =}\StringTok{ }\NormalTok{daily_rate }\OperatorTok\StringTok{ }
\StringTok{    }\KeywordTok{filter}\NormalTok{(group }\OperatorTok{==}\StringTok{ "b"}\NormalTok{) }\OperatorTok\StringTok{ }
\StringTok{    }\KeywordTok{summarize}\NormalTok{(}
      \DataTypeTok{meanrate =} \KeywordTok{mean}\NormalTok{(clickthroughrate)}
\NormalTok{    ) }\OperatorTok\StringTok{ }
\StringTok{    }\KeywordTok{pull}\NormalTok{ (meanrate)}
  
  \KeywordTok{return}\NormalTok{(}
    \KeywordTok{mean}\NormalTok{(mean_rate_a }\OperatorTok{-}\StringTok{ }\NormalTok{mean_rate_b)}
\NormalTok{  )}
\NormalTok{\}}

\NormalTok{bootstraps_q1 <-}\StringTok{ }\KeywordTok{boot}\NormalTok{(}\DataTypeTok{data =}\NormalTok{ events_q1bdif, }
                   \DataTypeTok{statistic =}\NormalTok{ funcao_bootstrap_q1_dif, }\CommentTok{# <- referência para a função }
                   \DataTypeTok{R =} \DecValTok{1000}\NormalTok{) }\CommentTok{# número de bootstraps}

\KeywordTok{glimpse}\NormalTok{(bootstraps_q1}\OperatorTok{$}\NormalTok{t)}
\end{Highlighting}
\end{Shaded}

\begin{verbatim}
##  num [1:1000, 1] 0.633 0.905 1.277 1.073 1.241 ...
\end{verbatim}

Vamos ver como ficou a distribuição amostral do parâmetro populacional
estimado via bootstrap. Estão marcadas no histograma duas linhas
verticais para os quantis 2,5\% e 97,5\%.

\begin{Shaded}
\begin{Highlighting}[]
\KeywordTok{tibble}\NormalTok{(}\DataTypeTok{parametro =} \KeywordTok{as.double}\NormalTok{(bootstraps_q1}\OperatorTok{$}\NormalTok{t)) }\OperatorTok\StringTok{ }
\StringTok{  }\KeywordTok{ggplot}\NormalTok{(}\KeywordTok{aes}\NormalTok{(}\DataTypeTok{x =}\NormalTok{ parametro)) }\OperatorTok{+}\StringTok{ }
\StringTok{  }\KeywordTok{geom_histogram}\NormalTok{(}\DataTypeTok{binwidth =} \FloatTok{0.1}\NormalTok{, }\DataTypeTok{fill =} \StringTok{"white"}\NormalTok{, }\DataTypeTok{color =} \StringTok{"blue"}\NormalTok{)}\OperatorTok{+}
\StringTok{  }\KeywordTok{geom_vline}\NormalTok{(}\DataTypeTok{xintercept =} \KeywordTok{quantile}\NormalTok{(bootstraps_q1}\OperatorTok{$}\NormalTok{t, }\FloatTok{0.025}\NormalTok{)[[}\DecValTok{1}\NormalTok{]]) }\OperatorTok{+}\StringTok{ }
\StringTok{  }\KeywordTok{geom_vline}\NormalTok{(}\DataTypeTok{xintercept =} \KeywordTok{quantile}\NormalTok{(bootstraps_q1}\OperatorTok{$}\NormalTok{t, }\FloatTok{0.975}\NormalTok{)[[}\DecValTok{1}\NormalTok{]])}
\end{Highlighting}
\end{Shaded}

\includegraphics{lab3-ckp1_files/figure-latex/unnamed-chunk-13-1.pdf}

Agora sim, vamos solicitar o cálculo dos intervalos de confiança que
contém o parâmetro calculado em 95\% das vezes que esse cálculo for
realizado.

\begin{Shaded}
\begin{Highlighting}[]
\KeywordTok{boot.ci}\NormalTok{(}\DataTypeTok{boot.out =}\NormalTok{ bootstraps_q1, }\DataTypeTok{conf =} \FloatTok{0.95}\NormalTok{, }\DataTypeTok{type =} \StringTok{"basic"}\NormalTok{)}
\end{Highlighting}
\end{Shaded}

\begin{verbatim}
## BOOTSTRAP CONFIDENCE INTERVAL CALCULATIONS
## Based on 1000 bootstrap replicates
## 
## CALL : 
## boot.ci(boot.out = bootstraps_q1, conf = 0.95, type = "basic")
## 
## Intervals : 
## Level      Basic         
## 95%   ( 0.309,  1.761 )  
## Calculations and Intervals on Original Scale
\end{verbatim}

Portanto, baseado nesse bootstrap, podemos afirmar com 95\% de confiança
que a média da taxa de cliques diária no grupo A está acima da taxa
média do grupo B num valor entre 13,28 e 14,18 pontos percentuais.

\subsection{PERGUNTA 2:}\label{pergunta-2}

\subsection{Quais resultados os usuários tendem a clicar primeiro? Como
isso muda dia após
dia?}\label{quais-resultados-os-usuarios-tendem-a-clicar-primeiro-como-isso-muda-dia-apos-dia}

Análise original.

\begin{Shaded}
\begin{Highlighting}[]
\NormalTok{events_q2 =}\StringTok{ }\NormalTok{events }\OperatorTok\StringTok{ }
\StringTok{    }\KeywordTok{filter}\NormalTok{(visit_index }\OperatorTok{==}\StringTok{ }\DecValTok{1} \OperatorTok{&}\StringTok{ }\NormalTok{action }\OperatorTok{==}\StringTok{ "visitPage"} \OperatorTok{&}\StringTok{ }\OperatorTok{!}\KeywordTok{is.na}\NormalTok{(result_position))}

\NormalTok{events_q2 }\OperatorTok\StringTok{ }
\StringTok{    }\KeywordTok{ggplot}\NormalTok{(}\KeywordTok{aes}\NormalTok{(}\DataTypeTok{x =}\NormalTok{ date_week, }\DataTypeTok{y =}\NormalTok{ result_position)) }\OperatorTok{+}
\StringTok{    }\KeywordTok{geom_jitter}\NormalTok{(}\DataTypeTok{alpha =} \FloatTok{0.05}\NormalTok{) }\OperatorTok{+}
\StringTok{    }\KeywordTok{labs}\NormalTok{(}\DataTypeTok{x =} \StringTok{"Dias"}\NormalTok{, }\DataTypeTok{y =} \StringTok{"Posição do resultado clicado"}\NormalTok{, }\DataTypeTok{title =} \StringTok{"Posição do primeiro clique por dia"}\NormalTok{) }\OperatorTok{+}
\StringTok{    }\KeywordTok{scale_y_log10}\NormalTok{() }\OperatorTok{+}
\StringTok{    }\KeywordTok{theme}\NormalTok{(}\DataTypeTok{axis.text.x =} \KeywordTok{element_text}\NormalTok{(}\DataTypeTok{angle=}\DecValTok{60}\NormalTok{, }\DataTypeTok{hjust=}\DecValTok{1}\NormalTok{))}
\end{Highlighting}
\end{Shaded}

\includegraphics{lab3-ckp1_files/figure-latex/Ploting data to answer question 2-1.pdf}

\begin{Shaded}
\begin{Highlighting}[]
\NormalTok{events_q2 }\OperatorTok
\StringTok{    }\KeywordTok{filter}\NormalTok{(result_position }\OperatorTok{<}\StringTok{ }\DecValTok{1000}\NormalTok{) }\OperatorTok\StringTok{ }
\StringTok{    }\KeywordTok{ggplot}\NormalTok{(}\KeywordTok{aes}\NormalTok{(}\DataTypeTok{x =}\NormalTok{ result_position)) }\OperatorTok{+}
\StringTok{    }\KeywordTok{geom_histogram}\NormalTok{(}\DataTypeTok{binwidth =} \DecValTok{1}\NormalTok{, }\DataTypeTok{fill =} \StringTok{"white"}\NormalTok{, }\DataTypeTok{color =} \StringTok{"blue"}\NormalTok{, }\DataTypeTok{show.legend =} \OtherTok{TRUE}\NormalTok{) }\OperatorTok{+}
\StringTok{    }\KeywordTok{labs}\NormalTok{(}\DataTypeTok{x =} \StringTok{"Posição do resultado clicado"}\NormalTok{, }\DataTypeTok{title =} \StringTok{"Posição do primeiro clique por dia"}\NormalTok{) }\OperatorTok{+}
\StringTok{    }\KeywordTok{scale_y_log10}\NormalTok{()}
\end{Highlighting}
\end{Shaded}

\begin{verbatim}
## Warning: Transformation introduced infinite values in continuous y-axis
\end{verbatim}

\begin{verbatim}
## Warning: Removed 132 rows containing missing values (geom_bar).
\end{verbatim}

\includegraphics{lab3-ckp1_files/figure-latex/Ploting data to answer question 2-2.pdf}

\subsection{Analisando a média de taxa de cliques diária
geral}\label{analisando-a-media-de-taxa-de-cliques-diaria-geral}

\begin{Shaded}
\begin{Highlighting}[]
\NormalTok{funcao_bootstrap_q2_global <-}\StringTok{ }\ControlFlowTok{function}\NormalTok{(data, indexes)\{}
\NormalTok{  daily_data =}\StringTok{ }\NormalTok{data }\OperatorTok\StringTok{ }
\StringTok{    }\KeywordTok{slice}\NormalTok{(indexes) }\OperatorTok\StringTok{ }
\StringTok{    }\KeywordTok{group_by}\NormalTok{(date_week) }\OperatorTok\StringTok{ }
\StringTok{    }\KeywordTok{summarize}\NormalTok{(}
      \DataTypeTok{mean_position =} \KeywordTok{mean}\NormalTok{(result_position)}
\NormalTok{    )}
  \KeywordTok{return}\NormalTok{(}
    \KeywordTok{mean}\NormalTok{(daily_data}\OperatorTok{$}\NormalTok{mean_position)}
\NormalTok{  )}
\NormalTok{\}}

\NormalTok{events_q2b =}\StringTok{ }\NormalTok{events_q2 }\OperatorTok
\StringTok{    }\KeywordTok{subset}\NormalTok{(}\DataTypeTok{select =} \KeywordTok{c}\NormalTok{(}\StringTok{"date_week"}\NormalTok{, }\StringTok{"result_position"}\NormalTok{))}


\NormalTok{bootstraps_q2b <-}\StringTok{ }\KeywordTok{boot}\NormalTok{(}\DataTypeTok{data =}\NormalTok{ events_q2b, }
                   \DataTypeTok{statistic =}\NormalTok{ funcao_bootstrap_q2_global, }\CommentTok{# <- referência para a função }
                   \DataTypeTok{R =} \DecValTok{1000}\NormalTok{) }\CommentTok{# número de bootstraps}

\KeywordTok{glimpse}\NormalTok{(bootstraps_q2b}\OperatorTok{$}\NormalTok{t)}
\end{Highlighting}
\end{Shaded}

\begin{verbatim}
##  num [1:1000, 1] 3.13 2.54 2.79 2.6 2.16 ...
\end{verbatim}

Vamos ver como ficou a distribuição amostral do parâmetro populacional
estimado via bootstrap. Estão marcadas no histograma duas linhas
verticais para os quantis 2,5\% e 97,5\%.

\begin{Shaded}
\begin{Highlighting}[]
\KeywordTok{tibble}\NormalTok{(}\DataTypeTok{parametro =} \KeywordTok{as.double}\NormalTok{(bootstraps_q2b}\OperatorTok{$}\NormalTok{t)) }\OperatorTok\StringTok{ }
\StringTok{  }\KeywordTok{ggplot}\NormalTok{(}\KeywordTok{aes}\NormalTok{(}\DataTypeTok{x =}\NormalTok{ parametro)) }\OperatorTok{+}\StringTok{ }
\StringTok{  }\KeywordTok{geom_histogram}\NormalTok{(}\DataTypeTok{binwidth =} \FloatTok{0.1}\NormalTok{, }\DataTypeTok{fill =} \StringTok{"white"}\NormalTok{, }\DataTypeTok{color =} \StringTok{"blue"}\NormalTok{)}\OperatorTok{+}
\StringTok{  }\KeywordTok{geom_vline}\NormalTok{(}\DataTypeTok{xintercept =} \KeywordTok{quantile}\NormalTok{(bootstraps_q2b}\OperatorTok{$}\NormalTok{t, }\FloatTok{0.025}\NormalTok{)[[}\DecValTok{1}\NormalTok{]]) }\OperatorTok{+}\StringTok{ }
\StringTok{  }\KeywordTok{geom_vline}\NormalTok{(}\DataTypeTok{xintercept =} \KeywordTok{quantile}\NormalTok{(bootstraps_q2b}\OperatorTok{$}\NormalTok{t, }\FloatTok{0.975}\NormalTok{)[[}\DecValTok{1}\NormalTok{]])}
\end{Highlighting}
\end{Shaded}

\includegraphics{lab3-ckp1_files/figure-latex/unnamed-chunk-16-1.pdf}

Agora sim, vamos solicitar o cálculo dos intervalos de confiança que
contém o parâmetro calculado em 95\% das vezes que esse cálculo for
realizado.

\begin{Shaded}
\begin{Highlighting}[]
\KeywordTok{boot.ci}\NormalTok{(}\DataTypeTok{boot.out =}\NormalTok{ bootstraps_q2b, }\DataTypeTok{conf =} \FloatTok{0.95}\NormalTok{, }\DataTypeTok{type =} \StringTok{"basic"}\NormalTok{)}
\end{Highlighting}
\end{Shaded}

\begin{verbatim}
## BOOTSTRAP CONFIDENCE INTERVAL CALCULATIONS
## Based on 1000 bootstrap replicates
## 
## CALL : 
## boot.ci(boot.out = bootstraps_q2b, conf = 0.95, type = "basic")
## 
## Intervals : 
## Level      Basic         
## 95%   ( 2.012,  3.716 )  
## Calculations and Intervals on Original Scale
\end{verbatim}

Portanto, baseado nesse bootstrap, podemos afirmar com 95\% de confiança
que os usários tendem a clicar nos resulados que estão entre as posições
3,173 e 4,124.

\subsection{PERGUNTA 3:}\label{pergunta-3}

\subsection{Qual a nossa taxa geral de pesquisas sem resultados
diariamente? Como isso varia entre os
grupos?}\label{qual-a-nossa-taxa-geral-de-pesquisas-sem-resultados-diariamente-como-isso-varia-entre-os-grupos}

Para responder esta questão precisamos contar a quantidade eventos do
tipo ``searchResultPage'' cuja coluna \emph{n\_result} é zero e dividir
pela quantidade total de eventos desse mesmo tipo.

O gráfico que segue é bem semelhante ao gráfico plotado na questão 1, só
muda a informação que estamos medindo, que aqui é a taxa de resultados
nulos, por assim dizer.

\begin{Shaded}
\begin{Highlighting}[]
\NormalTok{events_q3 =}\StringTok{ }\NormalTok{events }\OperatorTok
\StringTok{    }\KeywordTok{group_by}\NormalTok{(group, date_week) }\OperatorTok\StringTok{ }
\StringTok{    }\KeywordTok{summarise}\NormalTok{(}
        \DataTypeTok{zero_count =} \KeywordTok{sum}\NormalTok{(action }\OperatorTok{==}\StringTok{ "searchResultPage"} \OperatorTok{&}\StringTok{ }\NormalTok{n_results }\OperatorTok{==}\StringTok{ }\DecValTok{0}\NormalTok{),}
        \DataTypeTok{tot_count =} \KeywordTok{sum}\NormalTok{(action }\OperatorTok{==}\StringTok{ "searchResultPage"}\NormalTok{),}
        \DataTypeTok{zerorate =}\NormalTok{ zero_count}\OperatorTok{/}\NormalTok{tot_count}\OperatorTok{*}\DecValTok{100}
\NormalTok{    )}

\NormalTok{events_q3 }\OperatorTok\StringTok{ }
\StringTok{    }\KeywordTok{ggplot}\NormalTok{(}\KeywordTok{aes}\NormalTok{(}\DataTypeTok{x =}\NormalTok{ date_week, }\DataTypeTok{y =}\NormalTok{ zerorate, }\DataTypeTok{fill =}\NormalTok{ group)) }\OperatorTok{+}
\StringTok{    }\KeywordTok{geom_col}\NormalTok{(}\DataTypeTok{position =} \StringTok{"dodge"}\NormalTok{) }\OperatorTok{+}
\StringTok{    }\KeywordTok{labs}\NormalTok{(}\DataTypeTok{x =} \StringTok{"Dias"}\NormalTok{, }\DataTypeTok{y =} \StringTok{"Taxa de resultados nulos (%)"}\NormalTok{, }\DataTypeTok{fill =} \StringTok{"Grupos"}\NormalTok{, }\DataTypeTok{title =} \StringTok{"Taxa de pesquisas sem resultado por dia e por grupo"}\NormalTok{) }\OperatorTok{+}
\StringTok{    }\KeywordTok{theme}\NormalTok{(}\DataTypeTok{axis.text.x =} \KeywordTok{element_text}\NormalTok{(}\DataTypeTok{angle=}\DecValTok{60}\NormalTok{, }\DataTypeTok{hjust=}\DecValTok{1}\NormalTok{))}
\end{Highlighting}
\end{Shaded}

\includegraphics{lab3-ckp1_files/figure-latex/Preparing data to answer question 3-1.pdf}

\subsection{Analisando a média de taxa de pesquisas sem resultado geral
(sem distinção de
grupos)}\label{analisando-a-media-de-taxa-de-pesquisas-sem-resultado-geral-sem-distincao-de-grupos}

\begin{Shaded}
\begin{Highlighting}[]
\NormalTok{funcao_bootstrap_q3_global <-}\StringTok{ }\ControlFlowTok{function}\NormalTok{(data, indexes)\{}
\NormalTok{  daily_rate =}\StringTok{ }\NormalTok{data }\OperatorTok\StringTok{ }
\StringTok{    }\KeywordTok{slice}\NormalTok{(indexes) }\OperatorTok\StringTok{ }
\StringTok{    }\KeywordTok{group_by}\NormalTok{(date_week) }\OperatorTok\StringTok{ }
\StringTok{    }\KeywordTok{summarize}\NormalTok{(}
      \DataTypeTok{zero_rate =}\NormalTok{ (}\KeywordTok{sum}\NormalTok{(n_results}\OperatorTok{==}\DecValTok{0}\NormalTok{) }\OperatorTok{/}\StringTok{ }\KeywordTok{sum}\NormalTok{(n_results}\OperatorTok{>=}\DecValTok{0}\NormalTok{)) }\OperatorTok{*}\StringTok{ }\DecValTok{100}
\NormalTok{    )}
  \KeywordTok{return}\NormalTok{(}
    \KeywordTok{mean}\NormalTok{(daily_rate}\OperatorTok{$}\NormalTok{zero_rate)}
\NormalTok{  )}
\NormalTok{\}}

\NormalTok{events_q3b =}\StringTok{ }\NormalTok{events }\OperatorTok\StringTok{ }
\StringTok{    }\KeywordTok{group_by}\NormalTok{(date_week, session_id, search_index) }\OperatorTok\StringTok{ }
\StringTok{    }\KeywordTok{subset}\NormalTok{(action }\OperatorTok{==}\StringTok{ "searchResultPage"}\NormalTok{, }\DataTypeTok{select =} \KeywordTok{c}\NormalTok{(}\StringTok{"date_week"}\NormalTok{, }\StringTok{"n_results"}\NormalTok{))}


\NormalTok{bootstraps_q3 <-}\StringTok{ }\KeywordTok{boot}\NormalTok{(}\DataTypeTok{data =}\NormalTok{ events_q3b, }
                   \DataTypeTok{statistic =}\NormalTok{ funcao_bootstrap_q3_global, }\CommentTok{# <- referência para a função }
                   \DataTypeTok{R =} \DecValTok{1000}\NormalTok{) }\CommentTok{# número de bootstraps}

\KeywordTok{glimpse}\NormalTok{(bootstraps_q3}\OperatorTok{$}\NormalTok{t)}
\end{Highlighting}
\end{Shaded}

\begin{verbatim}
##  num [1:1000, 1] 19.3 19.4 20.8 19.1 20.7 ...
\end{verbatim}

Vamos ver como ficou a distribuição amostral do parâmetro populacional
estimado via bootstrap. Estão marcadas no histograma duas linhas
verticais para os quantis 2,5\% e 97,5\%.

\begin{Shaded}
\begin{Highlighting}[]
\KeywordTok{tibble}\NormalTok{(}\DataTypeTok{parametro =} \KeywordTok{as.double}\NormalTok{(bootstraps_q3}\OperatorTok{$}\NormalTok{t)) }\OperatorTok\StringTok{ }
\StringTok{  }\KeywordTok{ggplot}\NormalTok{(}\KeywordTok{aes}\NormalTok{(}\DataTypeTok{x =}\NormalTok{ parametro)) }\OperatorTok{+}\StringTok{ }
\StringTok{  }\KeywordTok{geom_histogram}\NormalTok{(}\DataTypeTok{binwidth =} \FloatTok{0.1}\NormalTok{, }\DataTypeTok{fill =} \StringTok{"white"}\NormalTok{, }\DataTypeTok{color =} \StringTok{"blue"}\NormalTok{)}\OperatorTok{+}
\StringTok{  }\KeywordTok{geom_vline}\NormalTok{(}\DataTypeTok{xintercept =} \KeywordTok{quantile}\NormalTok{(bootstraps_q3}\OperatorTok{$}\NormalTok{t, }\FloatTok{0.025}\NormalTok{)[[}\DecValTok{1}\NormalTok{]]) }\OperatorTok{+}\StringTok{ }
\StringTok{  }\KeywordTok{geom_vline}\NormalTok{(}\DataTypeTok{xintercept =} \KeywordTok{quantile}\NormalTok{(bootstraps_q3}\OperatorTok{$}\NormalTok{t, }\FloatTok{0.975}\NormalTok{)[[}\DecValTok{1}\NormalTok{]])}
\end{Highlighting}
\end{Shaded}

\includegraphics{lab3-ckp1_files/figure-latex/unnamed-chunk-19-1.pdf}

Agora sim, vamos solicitar o cálculo dos intervalos de confiança que
contém o parâmetro calculado em 95\% das vezes que esse cálculo for
realizado.

\begin{Shaded}
\begin{Highlighting}[]
\KeywordTok{boot.ci}\NormalTok{(}\DataTypeTok{boot.out =}\NormalTok{ bootstraps_q3, }\DataTypeTok{conf =} \FloatTok{0.95}\NormalTok{, }\DataTypeTok{type =} \StringTok{"basic"}\NormalTok{)}
\end{Highlighting}
\end{Shaded}

\begin{verbatim}
## BOOTSTRAP CONFIDENCE INTERVAL CALCULATIONS
## Based on 1000 bootstrap replicates
## 
## CALL : 
## boot.ci(boot.out = bootstraps_q3, conf = 0.95, type = "basic")
## 
## Intervals : 
## Level      Basic         
## 95%   (17.34, 21.27 )  
## Calculations and Intervals on Original Scale
\end{verbatim}

Portanto, baseado nesse bootstrap, podemos afirmar com 95\% de confiança
que a taxa de pesquisas sem resultado está entre 18,40\% e 18,83\%.

\subsection{Analisando a média de taxa de pesquisas sem resultados
diária geral (do grupo
A)}\label{analisando-a-media-de-taxa-de-pesquisas-sem-resultados-diaria-geral-do-grupo-a}

\begin{Shaded}
\begin{Highlighting}[]
\NormalTok{events_q3ba =}\StringTok{ }\NormalTok{events }\OperatorTok\StringTok{ }
\StringTok{    }\KeywordTok{filter}\NormalTok{(group }\OperatorTok{==}\StringTok{ "a"}\NormalTok{) }\OperatorTok\StringTok{ }
\StringTok{    }\KeywordTok{group_by}\NormalTok{(date_week, session_id, search_index) }\OperatorTok\StringTok{ }
\StringTok{    }\KeywordTok{subset}\NormalTok{(action }\OperatorTok{==}\StringTok{ "searchResultPage"}\NormalTok{, }\DataTypeTok{select =} \KeywordTok{c}\NormalTok{(}\StringTok{"date_week"}\NormalTok{, }\StringTok{"n_results"}\NormalTok{))}


\NormalTok{bootstraps_q3 <-}\StringTok{ }\KeywordTok{boot}\NormalTok{(}\DataTypeTok{data =}\NormalTok{ events_q3ba, }
                   \DataTypeTok{statistic =}\NormalTok{ funcao_bootstrap_q3_global, }\CommentTok{# <- referência para a função }
                   \DataTypeTok{R =} \DecValTok{1000}\NormalTok{) }\CommentTok{# número de bootstraps}

\KeywordTok{glimpse}\NormalTok{(bootstraps_q3}\OperatorTok{$}\NormalTok{t)}
\end{Highlighting}
\end{Shaded}

\begin{verbatim}
##  num [1:1000, 1] 20.8 21.7 22.2 21.5 20.6 ...
\end{verbatim}

Vamos ver como ficou a distribuição amostral do parâmetro populacional
estimado via bootstrap. Estão marcadas no histograma duas linhas
verticais para os quantis 2,5\% e 97,5\%.

\begin{Shaded}
\begin{Highlighting}[]
\KeywordTok{tibble}\NormalTok{(}\DataTypeTok{parametro =} \KeywordTok{as.double}\NormalTok{(bootstraps_q3}\OperatorTok{$}\NormalTok{t)) }\OperatorTok\StringTok{ }
\StringTok{  }\KeywordTok{ggplot}\NormalTok{(}\KeywordTok{aes}\NormalTok{(}\DataTypeTok{x =}\NormalTok{ parametro)) }\OperatorTok{+}\StringTok{ }
\StringTok{  }\KeywordTok{geom_histogram}\NormalTok{(}\DataTypeTok{binwidth =} \FloatTok{0.1}\NormalTok{, }\DataTypeTok{fill =} \StringTok{"white"}\NormalTok{, }\DataTypeTok{color =} \StringTok{"blue"}\NormalTok{)}\OperatorTok{+}
\StringTok{  }\KeywordTok{geom_vline}\NormalTok{(}\DataTypeTok{xintercept =} \KeywordTok{quantile}\NormalTok{(bootstraps_q3}\OperatorTok{$}\NormalTok{t, }\FloatTok{0.025}\NormalTok{)[[}\DecValTok{1}\NormalTok{]]) }\OperatorTok{+}\StringTok{ }
\StringTok{  }\KeywordTok{geom_vline}\NormalTok{(}\DataTypeTok{xintercept =} \KeywordTok{quantile}\NormalTok{(bootstraps_q3}\OperatorTok{$}\NormalTok{t, }\FloatTok{0.975}\NormalTok{)[[}\DecValTok{1}\NormalTok{]])}
\end{Highlighting}
\end{Shaded}

\includegraphics{lab3-ckp1_files/figure-latex/unnamed-chunk-22-1.pdf}

Agora sim, vamos solicitar o cálculo dos intervalos de confiança que
contém o parâmetro calculado em 95\% das vezes que esse cálculo for
realizado.

\begin{Shaded}
\begin{Highlighting}[]
\KeywordTok{boot.ci}\NormalTok{(}\DataTypeTok{boot.out =}\NormalTok{ bootstraps_q3, }\DataTypeTok{conf =} \FloatTok{0.95}\NormalTok{, }\DataTypeTok{type =} \StringTok{"basic"}\NormalTok{)}
\end{Highlighting}
\end{Shaded}

\begin{verbatim}
## BOOTSTRAP CONFIDENCE INTERVAL CALCULATIONS
## Based on 1000 bootstrap replicates
## 
## CALL : 
## boot.ci(boot.out = bootstraps_q3, conf = 0.95, type = "basic")
## 
## Intervals : 
## Level      Basic         
## 95%   (18.33, 23.70 )  
## Calculations and Intervals on Original Scale
\end{verbatim}

Portanto, baseado nesse bootstrap, podemos afirmar com 95\% de confiança
que a taxa de pesquisas sem resultado no grupo A da população está entre
18,28\% e 18,82\%.

\subsection{Analisando a média de taxa de pesquisas sem resultado diária
geral (do grupo
B)}\label{analisando-a-media-de-taxa-de-pesquisas-sem-resultado-diaria-geral-do-grupo-b}

\begin{Shaded}
\begin{Highlighting}[]
\NormalTok{events_q3bb =}\StringTok{ }\NormalTok{events }\OperatorTok\StringTok{ }
\StringTok{    }\KeywordTok{filter}\NormalTok{(group }\OperatorTok{==}\StringTok{ "b"}\NormalTok{) }\OperatorTok\StringTok{ }
\StringTok{    }\KeywordTok{group_by}\NormalTok{(date_week, session_id, search_index) }\OperatorTok\StringTok{ }
\StringTok{    }\KeywordTok{subset}\NormalTok{(action }\OperatorTok{==}\StringTok{ "searchResultPage"}\NormalTok{, }\DataTypeTok{select =} \KeywordTok{c}\NormalTok{(}\StringTok{"date_week"}\NormalTok{, }\StringTok{"n_results"}\NormalTok{))}


\NormalTok{bootstraps_q3 <-}\StringTok{ }\KeywordTok{boot}\NormalTok{(}\DataTypeTok{data =}\NormalTok{ events_q3bb, }
                   \DataTypeTok{statistic =}\NormalTok{ funcao_bootstrap_q3_global, }\CommentTok{# <- referência para a função }
                   \DataTypeTok{R =} \DecValTok{1000}\NormalTok{) }\CommentTok{# número de bootstraps}

\KeywordTok{glimpse}\NormalTok{(bootstraps_q3}\OperatorTok{$}\NormalTok{t)}
\end{Highlighting}
\end{Shaded}

\begin{verbatim}
##  num [1:1000, 1] 19.5 15.2 16.3 15.2 18.6 ...
\end{verbatim}

Vamos ver como ficou a distribuição amostral do parâmetro populacional
estimado via bootstrap. Estão marcadas no histograma duas linhas
verticais para os quantis 2,5\% e 97,5\%.

\begin{Shaded}
\begin{Highlighting}[]
\KeywordTok{tibble}\NormalTok{(}\DataTypeTok{parametro =} \KeywordTok{as.double}\NormalTok{(bootstraps_q3}\OperatorTok{$}\NormalTok{t)) }\OperatorTok\StringTok{ }
\StringTok{  }\KeywordTok{ggplot}\NormalTok{(}\KeywordTok{aes}\NormalTok{(}\DataTypeTok{x =}\NormalTok{ parametro)) }\OperatorTok{+}\StringTok{ }
\StringTok{  }\KeywordTok{geom_histogram}\NormalTok{(}\DataTypeTok{binwidth =} \FloatTok{0.1}\NormalTok{, }\DataTypeTok{fill =} \StringTok{"white"}\NormalTok{, }\DataTypeTok{color =} \StringTok{"blue"}\NormalTok{)}\OperatorTok{+}
\StringTok{  }\KeywordTok{geom_vline}\NormalTok{(}\DataTypeTok{xintercept =} \KeywordTok{quantile}\NormalTok{(bootstraps_q3}\OperatorTok{$}\NormalTok{t, }\FloatTok{0.025}\NormalTok{)[[}\DecValTok{1}\NormalTok{]]) }\OperatorTok{+}\StringTok{ }
\StringTok{  }\KeywordTok{geom_vline}\NormalTok{(}\DataTypeTok{xintercept =} \KeywordTok{quantile}\NormalTok{(bootstraps_q3}\OperatorTok{$}\NormalTok{t, }\FloatTok{0.975}\NormalTok{)[[}\DecValTok{1}\NormalTok{]])}
\end{Highlighting}
\end{Shaded}

\includegraphics{lab3-ckp1_files/figure-latex/unnamed-chunk-25-1.pdf}

Agora sim, vamos solicitar o cálculo dos intervalos de confiança que
contém o parâmetro calculado em 95\% das vezes que esse cálculo for
realizado.

\begin{Shaded}
\begin{Highlighting}[]
\KeywordTok{boot.ci}\NormalTok{(}\DataTypeTok{boot.out =}\NormalTok{ bootstraps_q3, }\DataTypeTok{conf =} \FloatTok{0.95}\NormalTok{, }\DataTypeTok{type =} \StringTok{"basic"}\NormalTok{)}
\end{Highlighting}
\end{Shaded}

\begin{verbatim}
## BOOTSTRAP CONFIDENCE INTERVAL CALCULATIONS
## Based on 1000 bootstrap replicates
## 
## CALL : 
## boot.ci(boot.out = bootstraps_q3, conf = 0.95, type = "basic")
## 
## Intervals : 
## Level      Basic         
## 95%   (13.22, 19.17 )  
## Calculations and Intervals on Original Scale
\end{verbatim}

Portanto, baseado nesse bootstrap, podemos afirmar com 95\% de confiança
que a taxa de pesquisas sem resultado no grupo B da população está entre
18,35\% e 19,12\%.

\subsection{Analisando a diferença da média de taxa de pesquisas sem
resultado diária entre o grupo A e o grupo
B}\label{analisando-a-diferenca-da-media-de-taxa-de-pesquisas-sem-resultado-diaria-entre-o-grupo-a-e-o-grupo-b}

\begin{Shaded}
\begin{Highlighting}[]
\NormalTok{events_q3bdif =}\StringTok{ }\NormalTok{events }\OperatorTok\StringTok{ }
\StringTok{    }\KeywordTok{group_by}\NormalTok{(group, date_week, session_id, search_index) }\OperatorTok\StringTok{ }
\StringTok{    }\KeywordTok{subset}\NormalTok{(action }\OperatorTok{==}\StringTok{ "searchResultPage"}\NormalTok{, }\DataTypeTok{select =} \KeywordTok{c}\NormalTok{(}\StringTok{"group"}\NormalTok{, }\StringTok{"date_week"}\NormalTok{, }\StringTok{"n_results"}\NormalTok{))}

\NormalTok{funcao_bootstrap_q3_dif <-}\StringTok{ }\ControlFlowTok{function}\NormalTok{(data, indexes)\{}
\NormalTok{  daily_rate =}\StringTok{ }\NormalTok{data }\OperatorTok\StringTok{ }
\StringTok{    }\KeywordTok{slice}\NormalTok{(indexes) }\OperatorTok\StringTok{ }
\StringTok{    }\KeywordTok{group_by}\NormalTok{(group, date_week) }\OperatorTok\StringTok{ }
\StringTok{    }\KeywordTok{summarize}\NormalTok{(}
      \DataTypeTok{zero_rate =}\NormalTok{ (}\KeywordTok{sum}\NormalTok{(n_results}\OperatorTok{==}\DecValTok{0}\NormalTok{) }\OperatorTok{/}\StringTok{ }\KeywordTok{sum}\NormalTok{(n_results}\OperatorTok{>=}\DecValTok{0}\NormalTok{)) }\OperatorTok{*}\StringTok{ }\DecValTok{100}
\NormalTok{    )}
  
\NormalTok{  mean_rate_a =}\StringTok{ }\NormalTok{daily_rate }\OperatorTok\StringTok{ }
\StringTok{    }\KeywordTok{filter}\NormalTok{(group }\OperatorTok{==}\StringTok{ "a"}\NormalTok{) }\OperatorTok\StringTok{ }
\StringTok{    }\KeywordTok{summarize}\NormalTok{(}
      \DataTypeTok{mean_rate =} \KeywordTok{mean}\NormalTok{(zero_rate)}
\NormalTok{    ) }\OperatorTok\StringTok{ }
\StringTok{    }\KeywordTok{pull}\NormalTok{ (mean_rate)}
    
\NormalTok{  mean_rate_b =}\StringTok{ }\NormalTok{daily_rate }\OperatorTok\StringTok{ }
\StringTok{    }\KeywordTok{filter}\NormalTok{(group }\OperatorTok{==}\StringTok{ "b"}\NormalTok{) }\OperatorTok\StringTok{ }
\StringTok{    }\KeywordTok{summarize}\NormalTok{(}
      \DataTypeTok{mean_rate =} \KeywordTok{mean}\NormalTok{(zero_rate)}
\NormalTok{    ) }\OperatorTok\StringTok{ }
\StringTok{    }\KeywordTok{pull}\NormalTok{ (mean_rate)}
  
  \KeywordTok{return}\NormalTok{(}
    \KeywordTok{mean}\NormalTok{(mean_rate_a }\OperatorTok{-}\StringTok{ }\NormalTok{mean_rate_b)}
\NormalTok{  )}
\NormalTok{\}}

\NormalTok{bootstraps_q3 <-}\StringTok{ }\KeywordTok{boot}\NormalTok{(}\DataTypeTok{data =}\NormalTok{ events_q3bdif, }
                   \DataTypeTok{statistic =}\NormalTok{ funcao_bootstrap_q3_dif, }\CommentTok{# <- referência para a função }
                   \DataTypeTok{R =} \DecValTok{1000}\NormalTok{) }\CommentTok{# número de bootstraps}

\KeywordTok{glimpse}\NormalTok{(bootstraps_q3}\OperatorTok{$}\NormalTok{t)}
\end{Highlighting}
\end{Shaded}

\begin{verbatim}
##  num [1:1000, 1] 1.23 3.15 1.86 5.43 6.06 ...
\end{verbatim}

Vamos ver como ficou a distribuição amostral do parâmetro populacional
estimado via bootstrap. Estão marcadas no histograma duas linhas
verticais para os quantis 2,5\% e 97,5\%.

\begin{Shaded}
\begin{Highlighting}[]
\KeywordTok{tibble}\NormalTok{(}\DataTypeTok{parametro =} \KeywordTok{as.double}\NormalTok{(bootstraps_q3}\OperatorTok{$}\NormalTok{t)) }\OperatorTok\StringTok{ }
\StringTok{  }\KeywordTok{ggplot}\NormalTok{(}\KeywordTok{aes}\NormalTok{(}\DataTypeTok{x =}\NormalTok{ parametro)) }\OperatorTok{+}\StringTok{ }
\StringTok{  }\KeywordTok{geom_histogram}\NormalTok{(}\DataTypeTok{binwidth =} \FloatTok{0.1}\NormalTok{, }\DataTypeTok{fill =} \StringTok{"white"}\NormalTok{, }\DataTypeTok{color =} \StringTok{"blue"}\NormalTok{)}\OperatorTok{+}
\StringTok{  }\KeywordTok{geom_vline}\NormalTok{(}\DataTypeTok{xintercept =} \KeywordTok{quantile}\NormalTok{(bootstraps_q3}\OperatorTok{$}\NormalTok{t, }\FloatTok{0.025}\NormalTok{)[[}\DecValTok{1}\NormalTok{]]) }\OperatorTok{+}\StringTok{ }
\StringTok{  }\KeywordTok{geom_vline}\NormalTok{(}\DataTypeTok{xintercept =} \KeywordTok{quantile}\NormalTok{(bootstraps_q3}\OperatorTok{$}\NormalTok{t, }\FloatTok{0.975}\NormalTok{)[[}\DecValTok{1}\NormalTok{]])}
\end{Highlighting}
\end{Shaded}

\includegraphics{lab3-ckp1_files/figure-latex/unnamed-chunk-28-1.pdf}

Agora sim, vamos solicitar o cálculo dos intervalos de confiança que
contém o parâmetro calculado em 95\% das vezes que esse cálculo for
realizado.

\begin{Shaded}
\begin{Highlighting}[]
\KeywordTok{boot.ci}\NormalTok{(}\DataTypeTok{boot.out =}\NormalTok{ bootstraps_q3, }\DataTypeTok{conf =} \FloatTok{0.95}\NormalTok{, }\DataTypeTok{type =} \StringTok{"basic"}\NormalTok{)}
\end{Highlighting}
\end{Shaded}

\begin{verbatim}
## BOOTSTRAP CONFIDENCE INTERVAL CALCULATIONS
## Based on 1000 bootstrap replicates
## 
## CALL : 
## boot.ci(boot.out = bootstraps_q3, conf = 0.95, type = "basic")
## 
## Intervals : 
## Level      Basic         
## 95%   ( 0.720,  8.527 )  
## Calculations and Intervals on Original Scale
\end{verbatim}

Portanto, baseado nesse bootstrap, podemos afirmar com 95\% de confiança
que a média da taxa de pesquisas sem resultado diária no grupo A não é
significativamente distinguível em relação à mesma medida do grupo B,
tendo em vista que a diferença oscila entre -0.6554 ponto percentual
(observar que trata-se de um limite inferior negativo) e +0,2904 ponto
percentual (positivo). Portanto, a hipótese de desse valores serem
iguais nos dois grupos é plausível e não pode ser descartada.

\section{Criando uma comparação A/A na pergunta
1}\label{criando-uma-comparacao-aa-na-pergunta-1}

\begin{Shaded}
\begin{Highlighting}[]
\NormalTok{events_q1aa =}\StringTok{ }\NormalTok{events }\OperatorTok\StringTok{ }
\StringTok{    }\KeywordTok{group_by}\NormalTok{(group, date_week, session_id, search_index) }\OperatorTok\StringTok{ }
\StringTok{    }\KeywordTok{mutate}\NormalTok{(}\DataTypeTok{click_count =} \KeywordTok{max}\NormalTok{(visit_index)) }\OperatorTok\StringTok{ }
\StringTok{    }\KeywordTok{subset}\NormalTok{(action }\OperatorTok{==}\StringTok{ "searchResultPage"} \OperatorTok{&}\StringTok{ }\NormalTok{group }\OperatorTok{==}\StringTok{ "a"}\NormalTok{, }\DataTypeTok{select =} \KeywordTok{c}\NormalTok{(}\StringTok{"group"}\NormalTok{, }\StringTok{"date_week"}\NormalTok{, }\StringTok{"click_count"}\NormalTok{))}

\CommentTok{#Gerando os índices da amostra A0 e da amostra A1}

\NormalTok{  indexes_a =}\StringTok{ }\KeywordTok{vector}\NormalTok{(}\DataTypeTok{mode =} \StringTok{"integer"}\NormalTok{, }\KeywordTok{nrow}\NormalTok{(events_q1aa))}
\NormalTok{  indexes_a[}\KeywordTok{sample}\NormalTok{(}\DecValTok{1}\OperatorTok{:}\KeywordTok{nrow}\NormalTok{(events_q1aa), }\DataTypeTok{size =} \KeywordTok{trunc}\NormalTok{(}\KeywordTok{nrow}\NormalTok{(events_q1aa)}\OperatorTok{/}\DecValTok{2}\NormalTok{), }\DataTypeTok{replace =} \OtherTok{FALSE}\NormalTok{)] =}\StringTok{ }\DecValTok{1}
\NormalTok{  indexes_a0 =}\StringTok{ }\KeywordTok{which}\NormalTok{(indexes_a }\OperatorTok{==}\StringTok{ }\DecValTok{0}\NormalTok{)}
\NormalTok{  indexes_a1 =}\StringTok{ }\KeywordTok{which}\NormalTok{(indexes_a }\OperatorTok{==}\StringTok{ }\DecValTok{1}\NormalTok{)}

\CommentTok{#Renomeando o grupo A para virar A0 e A1}
\NormalTok{events_q1aa}\OperatorTok{$}\NormalTok{group[indexes_a0] =}\StringTok{ "a0"}
\NormalTok{events_q1aa}\OperatorTok{$}\NormalTok{group[indexes_a1] =}\StringTok{ "a1"}

\NormalTok{funcao_bootstrap_q1_aa <-}\StringTok{ }\ControlFlowTok{function}\NormalTok{(data, indexes)\{}
\NormalTok{  daily_rate =}\StringTok{ }\NormalTok{data }\OperatorTok\StringTok{ }
\StringTok{    }\KeywordTok{slice}\NormalTok{(indexes) }\OperatorTok\StringTok{ }
\StringTok{    }\KeywordTok{group_by}\NormalTok{(group, date_week) }\OperatorTok\StringTok{ }
\StringTok{    }\KeywordTok{summarize}\NormalTok{(}
      \DataTypeTok{clickthrough_rate =}\NormalTok{ (}\KeywordTok{sum}\NormalTok{(click_count}\OperatorTok{>}\DecValTok{0}\NormalTok{) }\OperatorTok{/}\StringTok{ }\KeywordTok{sum}\NormalTok{(click_count}\OperatorTok{>=}\DecValTok{0}\NormalTok{)) }\OperatorTok{*}\StringTok{ }\DecValTok{100}
\NormalTok{    )}
\NormalTok{  mean_rate_a0 =}\StringTok{ }\NormalTok{daily_rate }\OperatorTok\StringTok{ }
\StringTok{    }\KeywordTok{filter}\NormalTok{(group }\OperatorTok{==}\StringTok{ "a0"}\NormalTok{) }\OperatorTok\StringTok{ }
\StringTok{    }\KeywordTok{summarize}\NormalTok{(}
      \DataTypeTok{mean_rate =} \KeywordTok{mean}\NormalTok{(clickthrough_rate)}
\NormalTok{    ) }\OperatorTok\StringTok{ }
\StringTok{    }\KeywordTok{pull}\NormalTok{ (mean_rate)}
    
\NormalTok{  mean_rate_a1 =}\StringTok{ }\NormalTok{daily_rate }\OperatorTok\StringTok{ }
\StringTok{    }\KeywordTok{filter}\NormalTok{(group }\OperatorTok{==}\StringTok{ "a1"}\NormalTok{) }\OperatorTok\StringTok{ }
\StringTok{    }\KeywordTok{summarize}\NormalTok{(}
      \DataTypeTok{mean_rate =} \KeywordTok{mean}\NormalTok{(clickthrough_rate)}
\NormalTok{    ) }\OperatorTok\StringTok{ }
\StringTok{    }\KeywordTok{pull}\NormalTok{ (mean_rate)}
  
  \KeywordTok{return}\NormalTok{(}
    \KeywordTok{mean}\NormalTok{(mean_rate_a0 }\OperatorTok{-}\StringTok{ }\NormalTok{mean_rate_a1)}
\NormalTok{  )}
\NormalTok{\}}

\NormalTok{bootstraps_q1aa <-}\StringTok{ }\KeywordTok{boot}\NormalTok{(}\DataTypeTok{data =}\NormalTok{ events_q1aa, }
                   \DataTypeTok{statistic =}\NormalTok{ funcao_bootstrap_q1_aa, }\CommentTok{# <- referência para a função }
                   \DataTypeTok{R =} \DecValTok{1000}\NormalTok{) }\CommentTok{# número de bootstraps}

\KeywordTok{glimpse}\NormalTok{(bootstraps_q1aa}\OperatorTok{$}\NormalTok{t)}
\end{Highlighting}
\end{Shaded}

\begin{verbatim}
##  num [1:1000, 1] 0.3849 0.2088 -0.0129 -0.8563 -0.5206 ...
\end{verbatim}

Vamos ver como ficou a distribuição amostral do parâmetro populacional
estimado via bootstrap. Estão marcadas no histograma duas linhas
verticais para os quantis 2,5\% e 97,5\%.

\begin{Shaded}
\begin{Highlighting}[]
\KeywordTok{tibble}\NormalTok{(}\DataTypeTok{parametro =} \KeywordTok{as.double}\NormalTok{(bootstraps_q1aa}\OperatorTok{$}\NormalTok{t)) }\OperatorTok\StringTok{ }
\StringTok{  }\KeywordTok{ggplot}\NormalTok{(}\KeywordTok{aes}\NormalTok{(}\DataTypeTok{x =}\NormalTok{ parametro)) }\OperatorTok{+}\StringTok{ }
\StringTok{  }\KeywordTok{geom_histogram}\NormalTok{(}\DataTypeTok{binwidth =} \FloatTok{0.1}\NormalTok{, }\DataTypeTok{fill =} \StringTok{"white"}\NormalTok{, }\DataTypeTok{color =} \StringTok{"blue"}\NormalTok{)}\OperatorTok{+}
\StringTok{  }\KeywordTok{geom_vline}\NormalTok{(}\DataTypeTok{xintercept =} \KeywordTok{quantile}\NormalTok{(bootstraps_q1aa}\OperatorTok{$}\NormalTok{t, }\FloatTok{0.025}\NormalTok{)[[}\DecValTok{1}\NormalTok{]]) }\OperatorTok{+}\StringTok{ }
\StringTok{  }\KeywordTok{geom_vline}\NormalTok{(}\DataTypeTok{xintercept =} \KeywordTok{quantile}\NormalTok{(bootstraps_q1aa}\OperatorTok{$}\NormalTok{t, }\FloatTok{0.975}\NormalTok{)[[}\DecValTok{1}\NormalTok{]])}
\end{Highlighting}
\end{Shaded}

\includegraphics{lab3-ckp1_files/figure-latex/unnamed-chunk-31-1.pdf}

Agora sim, vamos solicitar o cálculo dos intervalos de confiança que
contém o parâmetro calculado em 95\% das vezes que esse cálculo for
realizado.

\begin{Shaded}
\begin{Highlighting}[]
\KeywordTok{boot.ci}\NormalTok{(}\DataTypeTok{boot.out =}\NormalTok{ bootstraps_q1aa, }\DataTypeTok{conf =} \FloatTok{0.95}\NormalTok{, }\DataTypeTok{type =} \StringTok{"basic"}\NormalTok{)}
\end{Highlighting}
\end{Shaded}

\begin{verbatim}
## BOOTSTRAP CONFIDENCE INTERVAL CALCULATIONS
## Based on 1000 bootstrap replicates
## 
## CALL : 
## boot.ci(boot.out = bootstraps_q1aa, conf = 0.95, type = "basic")
## 
## Intervals : 
## Level      Basic         
## 95%   (-1.4838,  0.9401 )  
## Calculations and Intervals on Original Scale
\end{verbatim}

Portanto, baseado nesse bootstrap, podemos afirmar com 95\% de confiança
que a média da taxa de cliques diária dentro do próprio grupo A não
apresenta diferença significativa.


\end{document}
